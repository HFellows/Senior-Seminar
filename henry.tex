% This is a sample document using the University of Minnesota, Morris, Computer Science
% Senior Seminar modification of the ACM sig-alternate style. Much of this content is taken
% directly from the ACM sample document illustrating the use of the sig-alternate class. Certain
% parts that we never use have been removed to simplify the example, and a few additional
% components have been added.

% See https://github.com/UMM-CSci/Senior_seminar_templates for more info and to make
% suggestions and corrections.

\documentclass{sig-alternate}
\usepackage{color}
\usepackage[colorinlistoftodos]{todonotes}

\definecolor{Teal}{RGB}{2,132,130}
\newcommand{\allcomments}[1]{{#1}}
\newcommand{\hfcomment}[1]{\textcolor{Teal}{\allcomments{Henry: {#1}}}}

%%%%% Uncomment the following line and comment out the previous one
%%%%% to remove all comments
%%%%% NOTE: comments still occupy a line even if invisible;
%%%%% Don't write them as a separate paragraph
%\newcommand{\mycomment}[1]{}

\begin{document}

% --- Author Metadata here ---
%%% REMEMBER TO CHANGE THE SEMESTER AND YEAR AS NEEDED
\conferenceinfo{UMM CSci Senior Seminar Conference, December 2015}{Morris, MN}

\title{Modern Approaches to The (Rich?) Vehicle Routing Problem}

\numberofauthors{1}

\author{
% The command \alignauthor (no curly braces needed) should
% precede each author name, affiliation/snail-mail address and
% e-mail address. Additionally, tag each line of
% affiliation/address with \affaddr, and tag the
% e-mail address with \email
.
\alignauthor
Henry F. R. Fellows\\
	\affaddr{Division of Science and Mathematics}\\
	\affaddr{University of Minnesota, Morris}\\
	\affaddr{Morris, Minnesota, USA 56267}\\
	\email{fello056@morris.umn.edu}
}

\maketitle
\begin{abstract}
\end{abstract}
VRP


\keywords{VRP}

\section{Abstract}
\label{sec:abstract}
\hfcomment{Autonomous navigation is cool and important. Routing, being a part of Autonomous navigation, is a problem. It is hard. Good routing saves a lot of money and time.}
\section{Introduction}
\label{sec:intro}
\hfcomment{Routing is cool and important, also Amazon prime, Uber, self driving cars, and mail. Routing is NP-hard and part of combinatorial optimization problems, which include things like the traveling salesman problem.}
\subsection{Routing in the real world}
\label{ssec:real}
\hfcomment{Definition of ye olde Vehicle Routing Problem. Also there are a lot of variants; here are the ones I can find good sources about most common ones (a bit of sarcasm; they are actually the most researched ones, at least recently. A variant with dynamic nodes was very popular pre-2006.)}
\subsection{History of routing and problem?}
\section{The Vehicle Routing Problem}
\label{sec:VRP}
\hfcomment{Describe VRP in general terms, introduce RVRP.}
\subsection{Decentralized Vehicle Routing Problem}
\hfcomment{It's agent flavored!}
\subsection{Vehicle Routing Problem with Time Windows}
\hfcomment{Deliveries during business hours only!}
	
\section{Approaches}
\hfcomment{Generally, we use blackbox optimization, which is somewhat of a double-edged sword; it works well, and is computationally cheap(er), but it also does not produce any information about the data/problem it is working with. It is hard to gain insight into a problem given the output of a blackbox algorithm. This should maybe be a section for the following two items, but they're also the bulk (Guessing about 2/5ths) of my paper. Not sure what to do yet.}

\section{Genetic and Memetic Algorithms}
\hfcomment{Of Memetic Algorithms: In general, the genetic algorithm improves the solution in large strokes, while the local optimization fine tunes the solutions generated by the GA.}

\section{Agent-based models and Probability Collectives}
\label{sec:not this section.}
Many problems in the rich vehicle routing category feature decentralized control, where each truck or depot is modeled as having an independent agent who makes decisions out of self-interest. These decisions are often sub-optimal from the perspective of global optimization, and finding methods that allow for both autonomy and global utility is an ongoing research question. We have chosen to discuss two major approaches in modeling individuals - Agent-based models, the more traditional approach, and probability collectives, a relatively recent approach that has a legacy in genetic algorithms.
\subsection{Agent-based models}
\label{ssec:agents}
Agents are simply small decision making elements that typically have some internal state and the ability to make decisions based on this internal state. In optimization, a typical goal is for an agent to attempt to achieve some sort of optimal state. Agents (should) take actions to improve their state, and we describe this as the agent acting in its own interest. The function that returns the quantification of this local optimization is termed the local utility function, in contrast to the global, or system utility function. The global utility function describes how optimal the system is, which can be quite different from the local utility. In terms of the RVRP, the agent - usually a truck - attempts to find the best route possible without degrading the routes of other trucks. For example, it's possible to imagine a situation where a highly optimal set of destinations exists (with high local utility), but that if any truck were to choose it, that the remaining possible routes would be much less optimal (decreasing global utility).

\subsection{Probability Collectives}
\label{ssec:PC}
Probability Collectives (PC) is a formalism that shows game theory and statistical physics are identical in terms of information theory. Instead of attempting to evolve an exemplary individual in a population in the method of genetic/memetic algorithms, probability collectives selects optimal strategy for each agent. A PC agent is a self-interested, learning individual that desires to optimize it's own, local reward, while also optimizing the global, or system objectives. By including an inherent concern for the global objective, PC enables these agents to create a good outcome at the global level.

\subsubsection{Mechanics or something.}
\hfcomment{fix name, also RVRP = Rich Vehicle Routing Problem (e.g routing problem + more)}
The general form of PC in the context of VRP is as follows, along with the flowchart represented in figure 1.

	Imagine you have $N$ trucks, each having a set $X$ of actions, of (potentially variable) length $m$. The set is represented as $X_i=\{X_i^{[1]}, X_i^{[2]}, ..., X_i^{[m]}\}$. An action for truck $i$ is denoted as $X_i^{[r]}$, where $r$ is an identifier for that action. For each truck, assign a uniform probability, $1/m_i$ to all actions; the resulting probability of that action is signified by $q(X_i^{[r]})$. The truck then selects a random action $r$ and a sampling of random actions from other trucks. The resulting set, the 'combined strategy set', $Y_i^{[r]}=\{X_1^{[?]}, ...,X_i^{[r]}, ...,X_N^{[?]}\}$, \hfcomment{something}. For each of the strategy sets $Y_i^{[r]}$, compute the expected local utility - the measure of how good the solution is expected to be for the agent - using the following measure:
	\begin{equation}
	\textrm{Expected Utility of Agent } i^r =q_i^r\prod_{(i)}{q(x_{(i)}^{[?]})*G(Y_i^{[r]})}
	\end{equation}
Where $q_i^r$ represents the probability of action $r$ for truck $i$, $(i)$ is the set of all agents excluding $i$, and $G$ is the function that computes global utility for a given set of actions. $G$ is problem specific, but in the RVRP it would be something like a measure of unused capacity or unvisited destinations. After computing the local utility, the next step is to update the probability of each action for all trucks as follows:
	\begin{equation}
	q(X_i^{[r]})<-q(X_i^{[r]}-\alpha_{Step}*q(X_i^{[r]}*k_{\textit{r update}}
	\end{equation}
where $k$ is the iteration, and
	\begin{equation}
	k_{\textit{r update}} = (\textit{contribution of Agent }i)/T+S_i(q)+ln(q(X_i^{[r]}))
	\end{equation}
\hfcomment{T, the Boltzmann's temperature might be $T(k)=T_0/ln(k)$, but I can't find any good reference.}

and
	\begin{equation}
	\textit{contrib. of Agent }i = \textit{exp. utility of  Agent }i^r = \textit{exp. global utility}.
	\end{equation}	
\hfcomment{more to come once I understand it}


\section{Human Assist (tagging)}
\label{sec:humans}
\hfcomment{'Humans are really good at visually identifying good routes and tagging them accordingly. Also, server time is expensive and Amazon's Mechanical Turk is cheap.'}

\section{Conclusion}
\label{conclusion}

\section{Acknowledgements}
\hfcomment{Thanks to caffeine, water, and stress; the raw elements that formed this paper.}

\bibliography{henry}  
% You must have a proper ".bib" file
%  and remember to run:
% latex bibtex latex latex
% to resolve all references

\end{document}
