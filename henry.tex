% This is a sample document using the University of Minnesota, Morris, Computer Science
% Senior Seminar modification of the ACM sig-alternate style. Much of this content is taken
% directly from the ACM sample document illustrating the use of the sig-alternate class. Certain
% parts that we never use have been removed to simplify the example, and a few additional
% components have been added.

% See https://github.com/UMM-CSci/Senior_seminar_templates for more info and to make
% suggestions and corrections.

\documentclass{sig-alternate}
\usepackage{color}
\usepackage[colorinlistoftodos]{todonotes}

\definecolor{Teal}{RGB}{2,132,130}
\newcommand{\allcomments}[1]{{#1}}
\newcommand{\hfcomment}[1]{\textcolor{Teal}{\allcomments{Henry: {#1}}}}

%%%%% Uncomment the following line and comment out the previous one
%%%%% to remove all comments
%%%%% NOTE: comments still occupy a line even if invisible;
%%%%% Don't write them as a separate paragraph
%\newcommand{\mycomment}[1]{}

\begin{document}

% --- Author Metadata here ---
%%% REMEMBER TO CHANGE THE SEMESTER AND YEAR AS NEEDED
\conferenceinfo{UMM CSci Senior Seminar Conference, December 2015}{Morris, MN}

\title{Modern Approaches to The (Rich?) Vehicle Routing Problem}

\numberofauthors{1}

\author{
% The command \alignauthor (no curly braces needed) should
% precede each author name, affiliation/snail-mail address and
% e-mail address. Additionally, tag each line of
% affiliation/address with \affaddr, and tag the
% e-mail address with \email
.
\alignauthor
Henry F. R. Fellows\\
	\affaddr{Division of Science and Mathematics}\\
	\affaddr{University of Minnesota, Morris}\\
	\affaddr{Morris, Minnesota, USA 56267}\\
	\email{fello056@morris.umn.edu}
}

\maketitle
\begin{abstract}
\end{abstract}
VRP


\keywords{VRP}

\section{Abstract}
\label{sec:abstract}
\hfcomment{Autonomous navigation is cool and important. Routing, being a part of Autonomous navigation, is a problem. It is hard. Good routing saves a lot of money and time.}
\section{Introduction}
\label{sec:intro}
\hfcomment{Routing is cool and important, also Amazon prime, Uber, self driving cars, and mail. Routing is NP-hard and part of combinatorial optimization problems, which include things like the traveling salesman problem.}
\subsection{Routing in the real world}
\label{ssec:real}
\hfcomment{Definition of ye olde Vehicle Routing Problem. Also there are a lot of variants; here are the ones I can find good sources about most common ones (a bit of sarcasm; they are actually the most researched ones, at least recently. A variant with dynamic nodes was very popular pre-2006.)}
\subsection{History of routing and problem?}
\section{The Vehicle Routing Problem}
\label{sec:VRP}
\hfcomment{Describe VRP in general terms, introduce RVRP.}
\subsection{Decentralized Vehicle Routing Problem}
\hfcomment{It's agent flavored!}
\subsection{Vehicle Routing Problem with Time Windows}
\hfcomment{Deliveries during business hours only!}
	
\section{Approaches}
\hfcomment{Generally, we use blackbox optimization, which is somewhat of a double-edged sword; it works well, and is computationally cheap(er), but it also does not produce any information about the data/problem it is working with. It is hard to gain insight into a problem given the output of a blackbox algorithm. This should maybe be a section for the following two items, but they're also the bulk (Guessing about 2/5ths) of my paper. Not sure what to do yet.}

\section{Genetic and Memetic Algorithms}
\hfcomment{Of Memetic Algorithms: In general, the genetic algorithm improves the solution in large strokes, while the local optimization fine tunes the solutions generated by the GA.}

\section{Agent-based models and Probability Collectives}
\label{sec:not this section.}
Many problems in the rich vehicle routing category feature decentralized control, where each truck is modeled has having an independent agent who makes decisions out of self-interest. These decisions are often sub-optimal from the perspective of global optimization, and finding methods that allow for both autonomy and global utility is an ongoing research question. We have chosen to discuss two major approaches in modeling individuals - Agent-based models, the more traditional approach, and probability collectives, a relatively recent approach that has a legacy in genetic algorithms.
\subsection{Agent-based models}
\label{ssec:agents}

\subsection{Probability Collectives}
\label{ssec:PC}

\section{Human Assist (tagging)}
\label{sec:humans}
\hfcomment{'Humans are really good at visually identifying good routes and tagging them accordingly. Also, server time is expensive and Amazon's Mechanical Turk is cheap.'}

\section{Conclusion}
\label{conclusion}

\section{Acknowledgements}
\hfcomment{Thanks to caffeine, water, and stress; the raw elements that formed this paper.}

\bibliography{henry}  
% You must have a proper ".bib" file
%  and remember to run:
% latex bibtex latex latex
% to resolve all references

\end{document}
