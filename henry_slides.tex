\documentclass{beamer}

\mode<presentation>
{
  \usetheme{CambridgeUS}
  \setbeamercovered{transparent}
}

\usepackage{times}
\usepackage[T1]{fontenc} 
\usepackage{color}
\usepackage[colorinlistoftodos]{todonotes}
\usepackage{graphicx}
\usepackage[]{algorithm2e}
\usepackage{amssymb}
% Or whatever. Note that the encoding and the font should match. If T1
% does not look nice, try deleting the line with the fontenc.

\newcommand{\linespace}{\vskip 0.25cm}
\definecolor{MyForestGreen}{rgb}{0,0.7,0} 
\definecolor{Teal}{RGB}{2,132,130}
\newcommand{\allcomments}[1]{{#1}}
\newcommand{\hfcomment}[1]{\textcolor{Teal}{\allcomments{Henry: {#1}}}}


\title[Rich Vehicle Routing Problem]{Modern Approaches to The Rich Vehicle Routing Problem}
% Sub-titles are optional - uncomment and edit the next line if you want one.
% \subtitle{Why does sub-tree crossover work?} 
\author[Fellows]{Henry F. R. Fellows}
\institute[U of Minn, Morris]
{
  Division of Science and Mathematics \\
  University of Minnesota, Morris \\
  Morris, Minnesota, USA
}
% The text in square brackets is the short version of the date if you need that.
\date[???] % (optional)
{?? November 2016 \\ Computer Science Senior Seminar, Morris}

% Delete this, if you do not want the table of contents to pop up at
% the beginning of each subsection:
%\AtBeginSection[]{\begin{frame}<beamer>   \frametitle{Outline}    \tableofcontents[currentsection, hideothersubsections] \end{frame}}

\begin{document}

\begin{frame}
  \titlepage
\end{frame}

%outline
\begin{frame}
  \frametitle{Outline}
  \tableofcontents[hideallsubsections]
\end{frame}

% For a 20-25 minute senior seminar talk you probably want something like:
% - Two or three major sections (other than the summary).
% - At *most* three subsections per section.
% - Talk about 30s to 2min per frame. So there should probably be between
%   15 and 30 frames, all told.

\section{Routing in the real world}

\begin{frame}
\frametitle{Routing Irl!}
	\begin{itemize}
		\item The Post Office does routing!	
		\item Uber does routing!
		\item Routing is expensive!
	\end{itemize}
\end{frame}

\section{The Vehicle Routing Problem}

\begin{frame}
\frametitle{Traveling Salesman}
\end{frame}

\begin{frame}
\frametitle{Vehicle Routing Problem}
\end{frame}

\begin{frame}
\frametitle{Rich Vehicle Routing Problem}
\end{frame}


\subsection{Decentralized Vehicle Routing Problem}

\begin{frame}
\frametitle{Decentralized Vehicle Routing Problem}
\end{frame}

\subsection{Vehicle Routing Problem with Time Windows}

\begin{frame}
\frametitle{Vehicle Routing Problem with Time Windows}
\end{frame}

\section{Genetic and Memetic Algorithms} 

\begin{frame}
\frametitle{Genetic Algorithms I}
\end{frame}

\begin{frame}
\frametitle{Genetic Algorithms II}
\end{frame}

\begin{frame}
\frametitle{Memetic Algorithms}
\end{frame}

\subsection{Hybrid Genetic Search with Advanced Diversity Control}

\begin{frame}
\frametitle{Hybrid Genetic Search with Advanced Diversity Control}
\end{frame}

\begin{frame}
\frametitle{The algorithm, approximately.}
\scalebox{.5}{
\begin{algorithm}[H]
Initialize population\;
\While{number of interactions without improvement $< lt_{NI}$, and time $< T_{max}$}{
	Select parent solutions $P_1$ and $P_2$\;
	Create offspring $C$ from $P_1$ and $P_2$ (crossover)\;
	Educate $C$ (local search procedure)\;
	\If{$C$ infeasible}{
		Insert $C$ into infeasible subpopulation\;
		Repair with probability $P_rep$\;
	}
	\If{$C$ feasible}{
		Insert $C$ into feasible subpopulation\;
	}
	\If{maximum subpopulation size reached}{
		Select survivors\;
	}
	\If{best solution not improved for $It_{div}$ iterations}{
		Diversify population\;
	}
		Adjust penalty parameters for infeasibility\;
	\If{number of iterations $= k \times It_{dec}$ where $k \in \mathbb{N}$}{
		\hfcomment{k made up of natural numbers?}\;
		Decompose the master problem\;
		Use HGSADC on each subproblem\;
		Reconstitute three solutions, and insert them in the
		population\;
	}
}
\Return best feasible solution\;
\end{algorithm}}

Explanatory notes go here.
\end{frame}

\subsection{Another genetic algorithm-y technique}

\begin{frame}
\frametitle{Might not include in paper}
\end{frame}


\section{Agent-based models and Probability Collectives}

\subsection{Agent-based models}

\begin{frame}
\frametitle{Agents!}
\end{frame}

\subsection{Distributed Reverse Vickrey Auction Routing}

\begin{frame}
\frametitle{Reverse Vickrey Auction}
\end{frame}

\begin{frame}
\frametitle{Distributed RVA routing Algorithm}
\end{frame}

\subsection{Probability Collectives}

\begin{frame}
\frametitle{Probability Collectives}
\end{frame}


\begin{frame}
\frametitle{Probability Collectives Algorithm I}
\end{frame}

\begin{frame}
\frametitle{Probability Collectives Algorithm II}
\end{frame}

\section{Human Assisted Routing}

\begin{frame}
\frametitle{Human Assisted Routing}
\end{frame}

\section{Conclusions}

\begin{frame}
\frametitle{Routing Irl!}
	\begin{itemize}
		\item The HGSADC is the best.	
		\item Probability Collectives is best distributed system.
		\item Humans are still better than computers at guessing.
		\item Challenges remain in  routing with dynamic constraints.
	\end{itemize}
\end{frame}

%columns
\begin{frame}
  \frametitle{Column Example}
  \begin{columns}
  \begin{column}{.65\textwidth}
Same genome can lead to different physical structures or behavior depending on environmental factors.
  \end{column}
  \end{columns}
\end{frame}



\end{document}