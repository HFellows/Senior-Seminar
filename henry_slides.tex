\documentclass{beamer}

\mode<presentation>
{
  \usetheme{CambridgeUS}
  \setbeamercovered{transparent}
}

\usepackage{times}
\usepackage[T1]{fontenc} 
\usepackage{color}
\usepackage[colorinlistoftodos]{todonotes}
\usepackage{graphicx}
\usepackage[]{algorithm2e}
\usepackage{amssymb}
\usepackage{soul}
% Or whatever. Note that the encoding and the font should match. If T1
% does not look nice, try deleting the line with the fontenc.

\newcommand{\linespace}{\vskip 0.25cm}
\definecolor{MyForestGreen}{rgb}{0,0.7,0} 
\definecolor{Teal}{RGB}{2,132,130}
\newcommand{\allcomments}[1]{{#1}}
\newcommand{\hfcomment}[1]{\textcolor{Teal}{\allcomments{Henry: {#1}}}}


\title[Rich Vehicle Routing Problem]{Modern Approaches to The Rich Vehicle Routing Problem}
% Sub-titles are optional - uncomment and edit the next line if you want one.
% \subtitle{Why does sub-tree crossover work?} 
\author[Fellows]{Henry F. R. Fellows}
\institute[U of Minn, Morris]
{
  Division of Science and Mathematics \\
  University of Minnesota, Morris \\
  Morris, Minnesota, USA
}
% The text in square brackets is the short version of the date if you need that.
\date[???] % (optional)
{?? November 2016 \\ Computer Science Senior Seminar, Morris}

% Delete this, if you do not want the table of contents to pop up at
% the beginning of each subsection:
%\AtBeginSection[]{\begin{frame}<beamer>   \frametitle{Outline}    \tableofcontents[currentsection, hideothersubsections] \end{frame}}

\begin{document}

\begin{frame}
  \titlepage
\end{frame}

%outline
\begin{frame}
  \frametitle{Outline}
  \tableofcontents[hideallsubsections]
\end{frame}

% For a 20-25 minute senior seminar talk you probably want something like:
% - Two or three major sections (other than the summary).
% - At *most* three subsections per section.
% - Talk about 30s to 2min per frame. So there should probably be between
%   15 and 30 frames, all told.

\section{Routing in the real world}

\begin{frame}
\frametitle{What is routing?}
	definition of routing
\end{frame}

\begin{frame}
\frametitle{Routing Irl!}
	\begin{itemize}
		\item The Post Office does routing!	
		\item Uber does routing!
		\item Routing is expensive!
		\begin{itemize}
			\item Big!
			\item Scary!
			\item Numbers!
		\end{itemize}
	\end{itemize}
\end{frame}

\section{The Vehicle Routing Problem}

\begin{frame}
\frametitle{Traveling Salesman}
		\begin{itemize}
			\item "It's kinda the basic question of routing."
			\item Define.
			\item NP-hardness.
			\item NP-hard is hard!
		\end{itemize}
\end{frame}

\begin{frame}
\frametitle{Vehicle Routing Problem}
	\begin{itemize}
		\item History (verbal only.)
		\item Pretty graph.
		\item Description.
	\end{itemize}
\end{frame}

\begin{frame}
\frametitle{Rich Vehicle Routing Problem}
	\begin{itemize}
		\item Describe purpose \& intent
		\item Summarize.
	\end{itemize}
\end{frame}


\subsection{Decentralized Vehicle Routing Problem}

\begin{frame}
\frametitle{Decentralized Vehicle Routing Problem}
\end{frame}

\subsection{Vehicle Routing Problem with Time Windows}

\begin{frame}
\frametitle{Vehicle Routing Problem with Time Windows}
\end{frame}

\section{Genetic and Memetic Algorithms} 

\begin{frame}
\frametitle{Genetic Algorithms I}
	Loosely describe in a single slide. 
\end{frame}

\subsection{Hybrid Genetic Search with Advanced Diversity Control}

\begin{frame}
\frametitle{HGSADC}
	\begin{itemize}
		\item describe objectives
		\item describe techniques used
	\end{itemize}
\end{frame}

\begin{frame}
\frametitle{HGSADC Algorithm, loosely.}
 It's genuinely horrifying, so it might not actually get pseudocode. As I've gotten more familiar with it, I've actually found that it gives off a persistent feeling of brutality. 
\end{frame}

\begin{frame}
\frametitle{HGSADC Results}
\begin{itemize}
		\item Describe utter and ruthless dominance over everything else.
		\item Note: that even \st{Conan the barbarian} this algorithm can't solve every benchmark in the field.
		\item "To crush your enemies, to see them driven before you, and to hear the lamentations of their women!" - the goals of the algorithm authors.
		 
	\end{itemize}
\end{frame}

\section{Agent-based models and Probability Collectives}

\subsection{Agent-based models}

\begin{frame}
\frametitle{Agents}
	\begin{itemize}
		\item Agents definition
		\item example of agent based algorithm (conway's game?)
		\item hide picture of 007 on page.
	\end{itemize}
\end{frame}

\subsection{Distributed Reverse Vickrey Auction Routing}

\begin{frame}
\frametitle{Reverse Vickrey Auction}
	\begin{itemize}
		\item describe objectives
		\item describe economic rationale
	\end{itemize}
\end{frame}

\begin{frame}
\frametitle{Distributed RVA routing Algorithm}
	\begin{itemize}
		\item describe how it RVA is applied to routing. 
		\item note that much information is private.
	\end{itemize}

\end{frame}

\subsection{Probability Collectives}

\begin{frame}
\frametitle{Probability Collectives}
	\begin{itemize}
		\item Context
		\item Problem space difference from DRVA (it's a public-info algorithm; no hiding)		
	\end{itemize}
\end{frame}


\begin{frame}
\frametitle{Probability Collectives Algorithm I}
	It's worth going into detail for PC - it's easy and really, really cool.
\end{frame}

\begin{frame}
\frametitle{Probability Collectives Algorithm II}
\end{frame}

\section{Human Assisted Routing}

\begin{frame}
\frametitle{Human Assisted Routing}
	\begin{itemize}
		\item Describe role in routing (uses humans to find locally good routes.)
		\item Why is it useful (humans are cheap, O(n!) problem)
	\end{itemize}
\end{frame}

\section{Conclusions}

\begin{frame}
\frametitle{Conclusions}
	\begin{itemize}
		\item The HGSADC is \textit{the best}.	
		\item DRVA is best private distributed system		
		\item Probability Collectives is best public distributed system.
		\item Humans are still better than computers at guessing.
		\item Challenges remain in  routing with dynamic constraints.
	\end{itemize}
\end{frame}

%columns
\begin{frame}
  \frametitle{Column Example}
  \begin{columns}
  \begin{column}{.65\textwidth}
Same genome can lead to different physical structures or behavior depending on environmental factors.
  \end{column}
  \end{columns}
\end{frame}



\end{document}